\documentclass{mwart}
\usepackage[utf8]{inputenc}
\usepackage[MeX]{polski}
\usepackage{xcolor}
\usepackage{hyperref}
\usepackage{graphicx}
\usepackage{pdflscape}
\usepackage{pdfpages}

\title{Metody analizy sygnałów akustycznych generowanych podczas procesu spawania}
\author{Michał Dobrut}
\date{\today}

\begin{document}

\maketitle

\section{Metody spawania} 
Historia rozwoju praktycznych prac nad spawaniem sięga rewolucji przemysłowej, która opierała się w dużej mierze na wzroście kontroli nad procesami przetwarzania i łączenia metalu. Gwałtowny rozwój spawalnictwa zaczął podczas II Wojny Światowej ze względu na potrzeby wojskowe, a największe przełomy dokonały się w latach 1960-1970. Współcześnie wielkoskalowa produkcja wymaga głębokiej automatyzacji metod wytwarzanie wysokiej jakości połączeń elementów metalowych. Poniżej przybliżone (za \cite{naidu_modeling_2003}) są najpopularniejsze metody spawania wraz z podaniem funkcjonujących w literaturze skrótów i kodów.
\begin{enumerate}
    \item Spawanie elektrodami otulonymi (SMAW, MMA, met. 111) jest to proces spawanie łukowego w którym łuk utrzymywany między czubkiem elektrody pokrytej warstwą topnika a powierzchnią spawanego materiału generuje ciepło wystarczające do wytworzenia trwałego połączenia z wykorzystaniem dodawanego metalu z elektrody.
    \item Spawanie łukowe elektrodą nietopliwą w osłonie gazów obojętnych (GTAW, TIG, met. 141) jest procesem w którym do dostarczenia ciepła wykorzystywany jest łuk w osłonie gazowej pomiędzy trwałą, wolframową elektrodą a materiałem.
    \item Spawanie łukowe elektrodą topliwą w osłonie gazowej (Gas Metal Arc Welding (GMAW), MIG/MAG, met. 131/135) wykorzystuje łuk pomiędzy materiałem a wysuwanym na bieżąco drutem. Wyróżnia się 2 podgrupy metody w zależności od gazu osłonowego: Metal Active Gas dla CO$_2$ lub mieszaniny CO$_2$ z argonem i Metal Inert Gas dla argonu, helu lub mieszaniny.
    \item Spawanie drutem rdzeniowym (FCAW) jest procesem podobnym do GMAW, ale tu topnik znajduje się wewnątrz elektrody o kołowym przekroju. Może występować z dodatkową zewnętrzną osłoną gazową.
    \item Spawanie łukiem krytym (SAW) jest bardzo wydajną metodą w której łuk wytwarzany jest pomiędzy rozwijaną elektrodą (lub wieloma) a materiałem spawanym, przy czym łuk i jeziorko stopionego metalu są przykryte warstwą sypkiego topnika.
    \item Spawanie elektrogazowe (EGW) jest zautomatyzowanym procesem wykorzystywanym w konstrukcjach większej skali, gdzie rozwijane elektrody, o jednolitym przekroju lub z rdzeniem z topnika topią się dzięki utrzymaniu łuku z łączonymi powierzchniami, i wypełniają metalem przestrzeń pomiędzy chłodzonym ogranicznikami.
    \item Spawanie plazmowe (PAW) to proces, w którym łuk jest wytwarzany między materiałem spawanym a stałą elektrodą, schowaną w specjalnie ukształtowanej dyszy, ogniskującej strumień gazu w obszarze łuku. Może być wykorzystywane bez dodatkowego materiału wypełniającego.
\end{enumerate}

\section{Metoda GMAW}
\subsection{Opis metody}
Spawanie elektrodą topliwą w osłonie gazowej polega na ciągłym dodawaniu materiału pochodzącego z elektrody, odpowiedzialnej za wytwarzanie koniecznego ciepła, do jeziorka (rys. \ref{fig:schemat_gmaw}). 
\begin{figure}[ht]
    \centering
    \includegraphics[width=\linewidth]{GMAW_weld_area.png}
    \caption{Schemat spawania GMAW. 1 - kierunek spawania, 2 - końcówka prądowa, 3 – elektroda, 4 – gaz osłonowy, 5 – jeziorko,  6 - spoina, 7 – spawany materiał {\scriptsize Autor: Nathaniel C. Sheetz, \url{https://commons.wikimedia.org/wiki/File:GMAW_weld_area.png}}}
    \label{fig:schemat_gmaw}
\end{figure}
Stosuje się różne kombinacje gazów osłonowych, w tym najpopularniejsze:
\begin{itemize}
    \item czysty argon,
    \item 75/25\% lub 80/20\% argon-CO$_2$,
    \item argon + do 5\% tlenu dla spawania z dużymi prędkościami,
    \item 90\% He + 7.5\% Ar + 2.5\% CO$_2$ dla stali nierdzewnych.
\end{itemize}
Proces spawania GMAW charakteryzuje się dużą wydajnością i stosunkowo niskimi kosztami pracy oraz wymaga od spawacza umiarkowanych umiejętności. To przyczyniło się do jej szerokiego wykorzystania w warunkach produkcyjnych. Dobrze nadaje się do automatyzacji z użyciem robotów przemysłowych. Można nią łączyć szeroki zakres materiałów, w tym:
\begin{itemize}
    \item stal węglową,
    \item stal nierdzewną,
    \item stopy aluminum,
    \item stopy magnezu,
    \item miedź, brąz krzemowy,
    \item nikiel.
\end{itemize}
W przypadku transportu zwarciowego metoda jest ograniczona do cieńszych materiałów ale pozwala na spawanie w każdej pozycji. Transport natryskowy dostarcza większych ilości ciepła, ale ogranicza pozycję do podolnej. Metoda pulsacyjna eliminuje te wady, zmniejszając niestabilności łuku poprzez wprowadzenie wymuszonych oscylacji w proces, lecz wymaga bardziej zaawansowanego technologicznie sprzętu.    

\subsection{Transport materiału dodatkowego do jeziorka spawalniczego w procesie GMAW}
Można wyróżnić następujące sposoby transportu materiału elektrody do jeziorka spawalniczego\cite{naidu_modeling_2003,norrish_gas_2006, mizerski_spawanie_2013}:

\begin{enumerate}
    \item zwarciowy, zanurzeniowy (\emph{short-circuiting, dip}),
    \item lotny (\emph{free-flight}), w tym:
    \begin{enumerate}
        \item kropelkowy (\emph{globular}),
        \item natryskowy (\emph{spray}),
        \item strumieniowy (\emph{streaming}),
    \end{enumerate}
    \item pulsacyjny.
\end{enumerate}

Ich występowanie zależy od parametrów procesu spawania, przede wszystkim średnicy i materiału elektrody, szybkości posuwu drutu oraz prądu spawania. \cite{nadzam_gas_2014}

    \subsubsection{Transport zanurzeniowy}
    Gdy prędkość posuwu drutu jest wysoka, ciepło wytwarzane przez łuk nie jest wystarczające do stopienia drutu, przez co wchodzi on okresowo w kontakt z jeziorkiem. Następuje chwilowe zwarcie, prąd skacze do wartości zadanego ograniczenia, fragment drutu w pobliżu jeziorka jest stapiany i powraca łuk. Częstotliwość oscylacji wynosi 20-200~Hz, czas zwarcia kilka (ok.~4)~ms dla typowych parametrów.
    
    \subsubsection{Transport kropelkowy}
    Charakteryzuje się przekazywaniem materiału przez krople o~średnicy większej niż średnica elektrody i~częstotliwości spadania kropli rzędu kilku~Hz. Występuje przy średnich prądach wyższych niż w~transporcie zwarciowym, wystarczających do wydzielania w~łuku ciepła pozwalającego na bieżąco topić drut. 
    
    \subsubsection{Transport natryskowy}
    Prąd jeszcze wyższy niż w transporcie kropelkowym powoduje szybkie topienie materiału elektrody, który, w postaci kropli o średnicy zbliżonej do średnicy elektrody, jest wypychany przy znaczącym udziale sił elektromagnetycznych w stronę jeziorka. Dzięki temu możliwe jest spawanie w dowolnych pozycjach\cite{naidu_modeling_2003}. Występuje charakterystyczny kształt przetopu spoiny, z większym wtopieniem na osi ruchu rozpędzonego materiału elektrody.
    
    \subsubsection{Transport strumieniowych}
    Dla cienkich drutów o wysokim oporze i~prądach spawania wyższych niż dla transportu natryskowego obserwuje się strumień kropli o~średnicy mniejszej niż średnica elektrody i~powstawanie stożkowego zakończenia elektrody. Wysoka prędkość kropli nadawana przez siły elektromagnetyczne może powodować intensywny ruch w jeziorku i wprowadzenie pęcherzyków gazu\cite{norrish_gas_2006}.
    
    \subsubsection{Transport pulsacyjny}
    Na granicy pomiędzy transportem kropelkowym a natryskowym może ustalić się chwiejny stan równowagi charakteryzujący się małymi rozpryskami i~dużą wydajnością pracy\cite{norrish_gas_2006}. Stan ten wymaga precyzyjnego zasilania, jednak może wymuszony sztucznie poprzez pulsacyjną zmianę napięcia pomiędzy poziomami charakterystycznymi dla transportu kropelkowego (jako baza) i~natryskowego\cite{naidu_modeling_2003}.


\section{Opis fizyczny dźwięku}
Dźwięk jest poprzeczną falą mechaniczną rozchodzącą się w sprężystym ośrodku, nie zmieniając jego średniego przemieszczenia. Można go opisać kinematycznie jako formę względnych przemieszczeń cząsteczek ośrodka, czyli jako rodzaj zaburzenia, o określonym kierunku i prędkości przemieszczania się \cite{fahy_sound_2007}.
Podstawowymi pojęciami dla opisu dźwięku są\cite{sen_acoustics_nodate}:
\begin{itemize}
    \item \textbf{Okres $T$.} Czas pełnego cyklu oscylacji.
    \item \textbf{Częstotliwość $f$.} Liczba oscylacji punktu ośrodka sprężystego względem jego położenia średniego na sekundę.
    \item \textbf{Amplituda $A$.} Maksymalne odchylenie od średniej wartości położenia, lub odchylenie wartości ciśnienia od ciśnienia średniego, oznaczana jako P.
    \item \textbf{Faza $\varphi$.} Stan położenia punktu ośrodka w czasie i przestrzeni względem punktu odniesienia.
    \item \textbf{Prędkość dźwięku $\nu$.} Szybkość rozchodzenia się fali w ośrodku.
    \item \textbf{Długość fali $\lambda$.} Odległość jaką przebywa fala w kierunku propagacji w czasie jednego cyklu oscylacji opisana zależnością $\lambda = \frac{\nu}{f}$
\end{itemize}


W przypadku sinusoidalnej fali podłużnej ciśnienie akustyczne jest definiowane jako przemieszczające się zaburzenie ciśnienia o wartości
\begin{equation}
    p(t) = P sin(\omega t + \varphi).
\end{equation}
Dla tak opisanego dźwięku można określić parametr \textbf{ciśnienia efektywnego $p_{rms}$} zdefiniowanego jako
\begin{equation}
p_{rms}=\sqrt{\frac{1}{T}\int_0^T p^2(t)dt}
\end{equation} \cite{crocker_handbook_nodate}
Natężenie dźwięku $I$, jednostkę charakteryzującą moc przepływającą przez zamkniętą powierzchnię otaczającą źródło jest określana jako stosunek amplitudy $p$ do powierzchni. Dla fali płaskiej, czyli w przybliżeniu fali w dużej odległości od źródła punktowego, natężenie definiuje się następująco:
\begin{equation}
    I(r) = \frac{p_{rms}^2}{\rho c_0},
\end{equation} \cite{arata_investigation_1979}
gdzie $rho$ to gęstość ośrodka a $c_0$ to prędkość dźwięku w ośrodku.
Z tak określonego równania można wyliczyć moc punktowego źródła dźwięku emitującego w połówkę sfery za pomocą równania
\begin{equation}
    P = \frac{1}{2}  I_{śr}\, 4 \pi r^2 = \frac{p_{rms}^2}{\rho c_0}2 \pi r^2
\end{equation} \cite{crocker_handbook_nodate}
zakładając brak wnikania dźwięku w podłoże. Wynika z tego, że natężenie $I$ jest odwrotnie proporcjonalne do kwadratu odległości $r$ od źródła.
Znając relację natężenia do ciśnienia akustycznego, można także zauważyć, że ciśnienie jest odwrotnie proporcjonalne do odległości do źródła.
Poziom natężenia dźwięku $L$, jednostka biorąca pod uwagę charakter wrażliwości słuchu człowieka, jest definiowana jako
\begin{equation}
    L = 20\log_{10}\frac{p_{rms}}{p_{ref}}\, \mathrm{dB},\quad  \mathrm{gdzie}\quad p_{ref} = 2\cdot10^{-5} \, \mathrm{Pa}
\end{equation} \cite{crocker_handbook_nodate}.

\section{Generowanie dźwięku podczas spawania}
Dźwięk wytwarzany podczas spawania pochodzi z trzech źródeł: urządzeń spawalniczych, osłony gazowej i samego łuku elektrycznego. Zaobserwowano jednak, że natężenie dźwięku pochodzącego ze źródeł innych niż łuk jest znacznie niższe niż generowanego przez łuk\cite{schiebeck_audible_1991,saini_investigation_1998}, oraz jest pomijalne przy umieszczaniu mikrofonu blisko obszaru spawania\cite{luksa_przydatnosc_2012}.

Podczas analizy dźwięku generowanego przez łuk, dla odległości mikrofonu zbierającego niżej analizowane dane, można traktować go jako źródło punktowe, położone w pobliżu sztywnej powierzchni (spawanej blachy), emitujące energię w obszar o kształcie połówki sfery\cite{arata_investigation_1979}.

Dla spawania zwarciowego metodą GMAW, sygnał akustyczny ma charakter cyklicznych impulsów akustycznych  \cite{arata_investigation_1979, horvat_monitoring_2011} o średnim czasie trwania 3-6 ms\cite{luksa_przydatnosc_2012} mających charakter fali uderzeniowej \cite{arata_investigation_1979} i związanych z zajarzaniem i wygasaniem łuku pomiędzy zwarciami. Na to nałożony jest szum o charakterze turbulentnym pochodzący głównie od oscylacji łuku, elektrody i jeziorka oraz relaksacji naprężeń wewnątrz spawanego materiału\cite{horvat_monitoring_2011}.


Jak szczegółowo zbadano, wpływ na parametry sygnału dźwiękowego mają: 
\begin{enumerate}
\item moc wejściowa\cite{arata_investigation_1979, luksa_przydatnosc_2012},
\item szybkości spawania\cite{arata_investigation_1979},
\item wielkość przepływu gazu osłonowego\cite{arata_investigation_1979},
\item rodzaj gazu osłonowego\cite{arata_investigation_1979, luksa_przydatnosc_2012},
\item wysunięcie drutu z elektrody\cite{arata_investigation_1979},
\item rodzaj prostownika użytego w źródle zasilania\cite{luksa_przydatnosc_2012},
\item stan powierzchni spawanych elementów
\end{enumerate}
i inne czynniki.

Silna zależność dźwięku emitowanego w procesie spawania przez łuk od warunków zewnętrznych i parametrów pracy czyni go obiecującym źródłem informacji diagnostycznych do wykrywania wad powstałych podczas układania spoin.


\section{Analiza sygnału akustycznego na potrzeby diagnozowania błędów w procesie spawania}

Ze względu na silnie losowy charakter emitowanych dźwięków nie jest możliwa prosta ocena, czy proces spawania przebiegał poprawnie. W tym celu przy analizie sygnału akustycznego wykorzystuje się podzbiór jego cech punktowych i funkcyjnych dających dostatecznie dużo informacji najmniejszym nakładem obliczeniowym. Małe opóźnienie jest jest również pożądane w sytuacji monitorowania i korekcji procesu on-line.

\subsection{Badania dla procesu GMAW}

    Już w 1991r. Schiebeck\cite{schiebeck_audible_1991} zastanawiał się nad możliwością wykorzystania sygnału akustycznego do diagnozowania stabilności procesu spawania MAG, jednak oprócz porównania charakterystycznych cech empirycznych nie zaproponował użytecznego wskaźnika poza odchyleniem standardowym.
    
    Saini i Floyd \cite{saini_investigation_1998} zaproponowali i zbadali szereg cech sygnały akustycznego, konkludując, że częstotliwość przejść przez zero, pierwiastek z amplitudy i współczynnik kształtu pozwalają na wykrywanie odchyłek od idealnego zachowania łuku, a obiecujące są parametry domeny częstotliwościowej pod warunkiem dysponowania odpowiednią mocą obliczeniową.
    
    Grad i in.\cite{grad_feasibility_2004} zbadali możliwość monitorowania procesu MAG z pomocą sygnałów akustycznych i przetwornika drgań przymocowanego do spawanego materiału i stwierdzili przydatność cech sygnału, w szczególności kurtozy i dominującej częstotliwości w widmie mocy odpowiadającej transportowi kropli materiału, do oceny stabilności procesu
    
    Yusof i in. \cite{yusof_analysis_2014}stwierdzają (zbadana została metoda MiG) możliwość wykrycia z użyciem techniki rozkładu na mody empiryczne (Empirical Mode Decomposition, EMD) wystąpienia rozprysku lub oscylacji jeziorka, mogących prowadzić do powstania defektu. 

\section{Źródło danych wykorzystanych w pracy}
Niniejsza analiza bazuje na eksperymencie, w którym zebrane zostały dane z monitorowania zmiennych związanych z procesem spawania elementów z różnymi klasami uszkodzeń. Obserwowane były:
\begin{enumerate}
    \item napięcie i prąd spawania,
    \item prędkość spawania,
    \item natężenie przepływu gazu osłonowego,
    \item prędkość podawania drutu,
    \item emitowany sygnał akustyczny łuku spawalniczego
    \item obrazy wizyjne i termowizyjne łuku spawalniczego.
\end{enumerate}
\begin{figure}[ht]
    \centering
    \includegraphics[width=0.6\linewidth]{stanowisko.png}
    \caption{Caption}
    \label{fig:stanowisko}
\end{figure}
\begin{figure}[h!]
    \centering
    \includegraphics[width=\linewidth]{surowy_sygnal.png}
    \caption{Zarejestrowane dane dla 3 przykładowych pomiarów}
    \label{fig:3_runs}
\end{figure}
Stanowisko pomiarowe (rys. \ref{fig:stanowisko})
pozwalało na rejestrację procesu zautomatyzowanego spawania w pozycji podolnej procesem GMAW płyty o grubości 5mm drutem o średnicy 1.2~mm. Na potrzeby tej pracy wykorzystano wyłącznie oceny jakości połączeń oraz sygnał dźwiękowy zarejestrowany mikrofonem pola swobodnego 40BR firmy G.R.A.S. z osłoną przeciwwietrzną. Sygnał był próbkowany z częstotliwością 250~kHz, lecz niniejsze rozważania wykorzystują tylko informacje z zakresu słyszalnego. W badaniu zasymulowano najczęstsze przyczyny powstawanie niepoprawności w obrębie spoiny, a w szczególności:
\begin{enumerate}
    \item zmiany natężenia prądu i napięcia łuku,
    \item brak osłony gazowej,
    \item zużycie rolek podających drut,
    \item zanieczyszczenie powierzchni spawanej olejem,
    \item korozja drutu lub powierzchni spawanej,
    \item błędy ukosowania krawędzi spoiny oraz odstępu między płytami,
    \item zmiany prędkości spawania.
\end{enumerate}
Oprócz tego wykonano serię spoin bez wprowadzania zakłóceń.



\section{Cechy wybrane na potrzeby tej pracy}
\subsection{Kurtoza}
Z definicji:
\begin{equation}
    \mathrm{Kurt}= \frac{\mu_4}{\sigma^4}-3,
\end{equation}
gdzie $\mu_4$ to moment centralny 4. rzędu.
Parametr został wykorzystany ze względu na obserwowaną w wielkości ogonów (skrzydeł, wąsów) rozkłady różnicę względem rozkładu normalnego. Przewidywano, że większa burzliwość procesu w przypadku napotkania defektów materiału lub innych zaburzeń wpłynie na zwiększenie liczby obserwacji skrajnych wartości ciśnienia. Dodatkowo, parametr ten okazał się istotny w \cite{fidali_detection_2018}

\subsection{Entropia sygnału}
    Z definicji entropii dla dyskretnej zmiennej losowej $X$ o zbiorze wartości $\Omega = \{x_1, x_2, \dots , x_i\}$, 
    \begin{equation}
        H(X) = - \sum_i^N \left( p(x_i) \, log_2 p(xi) \right)
    \end{equation}co dla sygnału przekłada się na:
    \begin{equation}
        H(t) = -\sum_m^N \left( P(t, m) \, log_2P(t, m) \right)
    \end{equation}gdzie P(t, m) to moc m-tego binu częstotliwości znormalizowanego spektrum w chwili t.
    Parametr ten wybrano jako kolejną miarę losowości procesu która zmienia się w chwilach powstawania wady.
\subsection{Parametry rozkładu beta}
Rozkład beta jest zdefiniowany jako
\begin{equation}
    f(x, a, b) = \frac{x^{a-1}(1-x)^{b-1}}{\mathrm{B}\,(a, b)}\,,\quad gdzie \quad \mathrm{B}\, (a, b) = \frac{\Gamma(a)\Gamma(b)}{\Gamma(a+b)}
\end{equation}Korzystając ze związku parametrów $a$ i $b$ ze średnią $\mu$ i odchylenia standardowego $\sigma$ (metoda momentów), można znaleźć wartości parametrów bez skomplikowanych algorytmów dopasowujących krzywe.
\begin{equation}
a = \mu \left( \frac{\mu(1-\mu)}{\sigma^2} -1 \right)
\end{equation}
\begin{equation}
b = \frac{a(1-\mu)}{\mu}
\end{equation}
Parametry rozkładu beta użyto aby uwydatnić zmiany asymetrii  przebiegu sygnału. Niestety rozkład jest określony w przedziale [0:1], dlatego w skutek normalizacji okna tracona jest informacja o wartościach przyjmowanych przez sygnał w oknie.
\section{Przygotowanie danych}
Częstotliwość próbkowania sygnału akustycznego została obniżona metodą \'sinc\' z 250~kHz do 50~kHz oraz podzielona na pasma o szerokościach oktawy o częstotliwościach centralnych 16~Hz do 16~kHz zgodnie z normą IEC 61260:2014. Sygnał w każdym paśmie podzielono na okna o szerokości 8192 próbek (co przekłada się na ok. 164~ms), zachodzące na siebie w 50\%. Mniejsza szerokość okna uniemożliwia wnioskowanie dla niższych częstotliwości - musi być ona (czasowo) większa niż okres oscylacji sygnału w najniższym paśmie. Dla każdego okna obliczono wartości cech i przedstawiono je w razem za pomocą mapy barwnej.

Aby móc wnioskować o związku sygnału akustycznego z prądem i napięciem, naniesiono je na pomocniczy wykres: prąd został przepróbkowany na 500~Hz aby zwiększyć czytelność wykresu, napięcie ze względu na dużą zmienność przedstawiono po przypisaniu do jednego z 128 binów wartości rozłożonych równomiernie od wartości minimalnej do maksymalnej napięcia zarejestrowanej w czasie danego badania. Zaczernienie binu określa poglądowo względny czas spędzany w danym zakresie wartości.

\section{Wyniki analiz}
	\subsection{Brak wad}
	Obserwuje się dość znaczne skoki wartości kurtozy oraz parametrów rozkładu beta przy normalnym spawaniu, zwłaszcza dla wartości powyżej 1~kHz. Jak można było przewidywać, entropia jest wysoka dla pasm o wysokich częstotliwościach, gdzie znajduje się większość szumu, a niska w niskich. 
	\includepdf[landscape=true]{tex/brak_k_01_.pdf}
	
    \subsection{Korozja drutu}
    W prezentowanych cechach sygnału obserwuje się kilkukrotne zaburzenia procesu, objawiające się wzrostem wartości kurtozy i parametrów rozkładu beta oraz spadkiem entropii w pasmach powyżej ok. 200~Hz.
    W okolicach 9-10 sekundy nagrania widoczny jest odprysk od strony grani, z którym związane jest najbardziej wyróżniające się zaburzenie. 
    Obserwuje się współwystępowanie oscylacji prądu równocześnie z zaburzeniami w sygnale dźwiękowym.
	\includepdf[landscape=true]{tex/korozja_drut_k_02_.pdf}
	
	\subsection{Zmiana odstępu blach}
Obserwuje się wzrost gęstości pików wysokiej wartości kurtozy i parametrów a, b dla powiększania się rozstępu blach. Różnica jest widoczna na tym nagraniu, jednak porównując z nagraniem procesu spawania bez wad, nie da się zaobserwować znacznej różnicy w zakresie przyjmowanych wartości.

Zaburzenia  entropii widoczne w pasmach 125~Hz do 1~kHz w 11 do 15 sekundzie oraz ok. 27 sekundy są spowodowane zarejestrowaniem rozmów eksperymentatorów. Fakt ten świadczy o podatności systemu rejestrującego na zewnętrzne zaburzenia. Widać je również w postaci delikatnych cieni na przedstawieniu kurtozy i parametrów a, b.
	\includepdf[landscape=true]{tex/zmiana_odstep_blach_d_01_.pdf}
	
	\subsection{Zmiana prądu}
Zmiany prądu są zauważalne szczególnie dla entropii, szczególnie poprzez jej zwiększenie w paśmie średnich i wysokich częstotliwości, jesdak można zaobserwować pewne wygładzenie pozostałych cech także w tym obszarze związane ze zwiększeniem prądu.
	\includepdf[landscape=true]{tex/zmiana_I_d_01_.pdf}
	
	\subsection{Zmiana napięcia}
	Zmiany napięcia wyraźnie manifestują się w zarejestrowanym sygnale dźwiękowym: jego podwyższenie powoduje bardzo widoczne zwiększenie wartości cech poza entropią, która nieznacznie spada. Dotyczy to całego zakresu częstotliwości.
	
	Obniżenie napięcia powoduje oscylacje, które znacznie obniżają entropię w środkowym paśmie. Obserwuje się wystąpienie dwóch anomalii widocznych w parametrach kurtozy i a, b w okresie wzmożonych oscylacji prądu.
	\includepdf[landscape=true]{tex/zmiana_U_d_01_.pdf}
	
	\subsection{Korozja blachy}
	Nie zaobserwowano znaczącej różnicy pomiędzy obszarami skorodowanymi i wolnymi od korozji.
	\includepdf[landscape=true]{tex/kor_blacha_k_01_.pdf}
	
	\subsection{Zmiana ukosowania}
	Zaobserwowano niewielki wzrost kurtozy oraz wartości parametrów a, b oraz niewielki spadek entropii w średnim paśmie wraz ze zwiększaniem kąta rowka.
	\includepdf[landscape=true]{tex/zmiana_ukosowanie_d_01_.pdf}
	
	\subsection{Otwory w ukosowaniu}
	Nie udało się stwierdzić występowania anomalii w cechach sygnału związanych z obecnością otworów.
	\includepdf[landscape=true]{tex/ukosowanie_otwory_d_01_.pdf}
	
	\subsection{Zabrudzenie olejem}
	Obszar z błędami w spoinie pokrywa się z wyraźnym podwyższeniem wartości kurtozy i a, b orz spadkiem entropii, jednak zauważono, że nieprawidłowości spoiny zaczynają się przed wystąpieniem zaburzeń sygnału i kończą po ich ustąpieniu.
	
	Dodatkowo pod koniec nagrania zaobserwowano mniejszą anomalię, niewidoczną na spoinie.
	\includepdf[landscape=true]{tex/zabr_olej_k_01_.pdf}
	
	\subsection{Zmiana prędkości posuwu}
	W obszarze przyśpieszenia posuwu wyraźnie widać spadek entropii oraz nieznaczny wzrost wartości pozostałych cech, szczególnie kurtozy. Wraz ze zmianą prędkości posuwu zmieniano nieco napięcie (+/- 15\%) dlatego ciężko jednoznacznie określić przyczynę występowania zaburzeń sygnału.
	\includepdf[landscape=true]{tex/zmiana_V_d_01_.pdf}

	\subsection{Zużycie rolki podające drut}
	Zużycie rolki powoduje nierównomierną prędkość podawania drutu, co powinno być widoczne na spoinie. Błędy te są szczególnie widoczne na spoinie po 30. sekundzie i objawiają się nieznacznym spadkiem entropii oraz wzrostem wartości pozostałych cech w paśmie średnim i wysokim. 
	
	Obserwowane oscylacje napięcia w przedziale czasu od 21. do 28. sekundy widoczne są jako nieznaczne wyrównanie wartości (rozmycie, cień) cech w zakresie średnich i wysokich częstotliwości.
	\includepdf[landscape=true]{tex/zuz_rolki_k_01_.pdf}
	
	\subsection{Brak gazu osłonowego}
	Widoczne anomalie w entropii  w średnim zakresie częstotliwości są nagraniem komend eksperymentatora dotyczących włączania i wyłączania przepływu gazu, jednak nie zaobserwowano związanych z tym zmian na spoinie. Nie udało się powiązać stanu przepływu gazu z obrazem cech.
	\includepdf[landscape=true]{tex/brak_gaz_k_01_.pdf}
	
	\subsection{Wnioski końcowe}
	Zarejestrowany sygnał, po przetworzeniu w celu uzyskania wartości interesujących cech, charakteryzuje się sporym szumem, który ukrywa anomalie sygnalizujące wystąpienie wad, jednak jest są one w niektórych przypadkach możliwe do oceny, szczególnie zmiana napięcia i prądu spawania oraz korozja drutu były wyraźnie zauważalne. Brak znajomości dokładnego miejsca wystąpienia wad (patrz: zabrudzenie olejem, brak gazu) powoduje trudność w klasyfikacji które zaburzenia sygnału są decydujące przy przewidywaniu jakości spoiny.
	
	Nie wyciągnięto żadnych dodatkowych informacji z parametru a względem parametru b rozkładu beta, dlatego proponuje się zrezygnować z nich, pozostawiając jedynie kurtozę, jako że obserwacje dla tych 3 cech były najczęściej identyczne. Proponuje się wprowadzić w ich miejsce skośność aby lepiej uchwycić asymetrię rozkładu.
	


\bibliographystyle{plain} 
\bibliography{references}

\end{document}
